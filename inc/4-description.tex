\section{Описание виртуальной инфраструктуры}

Пункты:?
\begin{itemize}
  \item Назначение
  \item Системные требования, парк серверов
  \item Технологии, ПО, библиотеки, скрипты
  \item Алгоритмы (несколько, они довольно сложные) функционирования инфраструктуры 
\end{itemize}

Ключевые слова:
виртуализация,
KVM,
OpenVZ,
выделенный сервер,
VPS,
мониторинг,
nagios,
munin,
резервное копирование,
полный/инкрементальный/дифференциальный бэкапы,
репликация MySQL,
репликация DNS,
шардинг,
CDN,
типы репликации DNS и MySQL,
DDoS и защита от него,
LVM,
RAID,
панель управления,
ISPmanager/Vesta/Plesk/cpanel/ajenti...,
ISPsystem и его продукция,
обоснование выбора OpenVZ и KVM,
работа с ДЦ,
лицензии на ПО и подсети IP с арендуемым железом,
биллинг (платежная система),
скрипты самопальные,
свои конфиги,
тестирование хостинг-панелей для клиентов,
клиентская и админская документация,
тарифы (услуги),
миграция контейнеров и серверов,
отказоустойчивость,
расширение инфраструктуры

\clearpage
