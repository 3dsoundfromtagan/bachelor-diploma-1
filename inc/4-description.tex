\section{Описание виртуальной инфраструктуры}

\iffalse 
Пункты:?
\begin{itemize}
  \item Назначение
  \item Системные требования, парк серверов
  \item Технологии, ПО, библиотеки, скрипты
  \item Алгоритмы (несколько, они довольно сложные) функционирования инфраструктуры 
\end{itemize}

Ключевые слова:
виртуализация,
KVM,
OpenVZ,
выделенный сервер,
VPS,
мониторинг,
nagios,
munin,
резервное копирование,
полный/инкрементальный/дифференциальный бэкапы,
репликация MySQL,
репликация DNS,
шардинг,
CDN,
балансировка нагрузки,
типы репликации DNS и MySQL,
DDoS и защита от него,
LVM,
RAID,
панель управления,
ISPmanager/Vesta/Plesk/cpanel/ajenti...,
ISPsystem и его продукция,
обоснование выбора OpenVZ и KVM,
работа с ДЦ,
лицензии на ПО и подсети IP с арендуемым железом,
биллинг (платежная система),
скрипты самопальные,
свои конфиги,
тестирование хостинг-панелей для клиентов,
клиентская и админская документация,
тарифы (услуги),
миграция контейнеров и серверов,
отказоустойчивость,
расширение инфраструктуры

Ссылки: http://blog.selectel.ru/balansirovka-nagruzki-osnovnye-algoritmy-i-metody/

Рисую в гугл драйве схемы.
\fi 

Виртуальная инфраструктура --- это частное облако, размещаемое на оборудовании провайдера.
По сути, виртуальная инфраструктура представляет собой динамическое распределение ресурсов в соответствии предприятия.
Виртуальная машина использует материальные ресурсы одного компьютера, а виртуальная инфраструктура --- материальные ресурсы всей ИТ-среды, формируя из компьютеров, а также из подключенных к ним сетей и хранилищ единый пул ИТ-ресурсов.

Виртуальная инфраструктура включает в себя следующие компоненты \cite{virt-infrast}:
\begin{itemize}
  \item Гипервизоры для одного узла для полной виртуализации каждого компьютера;
  \item Пакет услуг инфраструктуры распределенных систем на основе виртуализации (например, управление ресурсами) для оптимального распределения доступных ресурсов между виртуальными машинами;
  \item Решения для автоматизации, обеспечивающие особые возможности оптимизации того или иного ИТ-процесса (например инициализации или восстановления в критических ситуациях).
\end{itemize}

Благодаря отделению всей программной среды от исходной аппаратной инфраструктуры виртуализация позволяет объединить ряд серверов, инфраструктур хранения и сетей в единый пул ресурсов, динамически, безопасно и надежно распределяемый между приложениями по мере необходимости.
С помощью этого инновационного решения организации могут создать вычислительную инфраструктуру с максимальной эффективностью, доступностью, автоматизацией и гибкостью, состоящую из недорогих серверов, соответствующих отраслевому стандарту.

Предпочтение виртуальной инфраструктуре отдают по причинам:
\begin{itemize}
  \item Экономии на обслуживающем персонале при условии сохранения 100\% отказоустойчивости системы;
  \item Отсутствии необходимости выделять бюджет на модернизацию оборудования;
  \item Возможности объединения в общую виртуальную среду офисы, географически находящиеся в разных местах;
  \item Возможности быстрого масштабирования проекта в соответствии с текущими решаемыми задачами;
  \item Быстрого доступа к данным.
\end{itemize}

Под каждый тарифный план создается изолированное частное облако с фиксированным количеством выделенных ресурсов.
Выделенное в рамках тарифного плана частное облако становится гибкой виртуальной оболочкой.
Функционирующие внутри нее виртуальные машины могут объединяться в виртуальные сети.
Изменение их вычислительной мощности происходит в зависимости от решаемых задач.
Смена или трансформация тарифных планов производится в режиме реального времени без перебоев в работе.

В случае необходимости, пользователь имеет возможность создавать нужное количество сегментов сети, добиваясь удобной конфигурации виртуальной рабочей среды.

Управление ресурсами осуществляется через удобный для пользователя веб-интерфейс, что позволяет делать все необходимые операции, такие как:
\begin{itemize}
  \item Создание арендуемых виртуальных серверов самостоятельно;
  \item Изменение конфигурации за несколько минут, часть операций даже без остановки и перезагрузки сервера (для некоторых операционных систем);
  \item Включение, выключение, установка, переустановка ОС и приложений самостоятельно и удаленно;
  \item Осуществление резервного копирование и сохранение состояний работающих виртуальных машин. 
\end{itemize}

\subsection{Назначение виртуальной инфраструктуры}

\clearpage
