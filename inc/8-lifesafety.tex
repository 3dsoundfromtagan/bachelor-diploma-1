\section{Безопасность жизнедеятельности}

В разделе выполнен анализ условий труда администратора и их требованиям освещенности, электробезопасности, микроклимата, шума, требованиям к оборудованию, к организации рабочего места пользователя ПК.

\subsection{Краткая характеристика помещения}

Согласно санитарным нормам, ширина стола должна быть не менее 0.6~м, глубина не менее 0.8~м, также необходимо обеспечить расстояние между боковыми поверхностями мониторов не менее 1.2~м.

На рис. \ref{roomplan} изображена планировка и размещение оборудования на рабочем месте.
\addimghere{roomplan}{0.85}{Планировка и размещение оборудования на рабочем месте \\
1 --- дверь; 2 --- тумбочка; 3 --- холодильник; 4 --- стол; 5 --- кресло; 6 --- ноутбук; 7 --- монитор; 8 --- планшет; 9 --- кровать; 10 --- окно; 11 --- шкаф; 12 --- обеденный стол}{roomplan}

\subsection{Микроклимат}

Метеорологические условия (микроклимат) характеризуются следующими параметрами:
\begin{itemize}
  \item температурой воздуха;
  \item относительной влажностью;
  \item скоростью движения воздуха на рабочем месте;
  \item барометрическим давлением.
\end{itemize}

В комнате, где находится персональный компьютер, должен поддерживаться определенный температурный режим для нормальной эксплуатации ЭВМ и условий труда администратора.

Параметры воздушной среды в кабинете должны соответствовать требованиям ГОСТ 12.1.005-88 <<Общие санитарно-гигиенические требования к воздуху санитарной зоны>>.
При этом необходимо учитывать, что работа администратора относится к разряду легких работ (разряд 1а --- затраты энергии до 150 ккал/час), а кабинет --- к производственным помещениям.

В помещении имеются источники избыточного тепла:
\begin{itemize}
  \item тепловыделение от ноутбука;
  \item тепло, выделяемое персоналом.
\end{itemize}

Тепловыделение от светильников отсутствует, так как используется люминесцентное освещение.
Для поддержания оптимального уровня температуры используется как естественная вентиляция (через двери и окна путем проветривания), так и искусственная (при помощи вентилятора), так как в текущих условиях проживания невозможно установить кондиционер.
Для обогрева помещения в зимнее время используется водяное отопление.

\subsection{Освещение}

В зависимости от необходимости, производственное освещение в кабинете может быть как естественным, создаваемым непосредственно солнцем и диффузным (рассеянным) светом, так и искусственным, осуществляемым электрическими лампами.
Естественное освещение характеризуется тем, что создаваемое освещение изменяется в очень широких пределах, в зависимости от времени года, дня и метеорологических факторов.
При выборе норм естественного освещения учитывается разряд выполняемых работ, система освещения, коэффициент солнечности, коэффициент светового климата \cite{oot}.

Мероприятия, за счет которых выполняются требования норм СП 52.13330.2011 <<Естественное и искусственное освещение>>:
\begin{itemize}
  \item проверка, не реже одного раза в год, соответствия освещенности на рабочей поверхности нормам искусственного освещения;
  \item очистка светильников не реже одного раза в три месяца;
  \item протирка окон (стекол) не реже двух раз в год.
\end{itemize}

\subsection{Шум и вибрация}

Источниками шума в помещении являются:
\begin{itemize}
  \item непосредственно персональный компьютер (вентиляторы охлаждения процессора и видеокарты);
  \item разговорная речь;
  \item шум вне рабочей зоны.
\end{itemize}

Постоянный шум оказывает отрицательное воздействие на человека, как биологически, так и психологически, что отражается на качестве работы и общей производительности труда сотрудников.
Снижается производительность труда и повышается количество допущенных ошибок, некоторые из которых могут быть критическими.
Допустимый уровень звука --- 50~дБА, при работающем оборудовании в кабинете ожидаемый уровень звука --- 40-48~дБА.

\subsection{Пожаробезопасность}

Пожар в помещении может возникнуть при взаимодействии горючих веществ, окислителя (условия пожара) и источников воспламенения (причина пожара).
Горючие вещества в кабинете: деревянные столы, двери, полы (паркет), покрытия стен, изоляция соединительных кабелей, жидкости для протирки узлов компьютера и другие.

Возможные источники и причины возникновения пожара:
\begin{itemize}
  \item эксплуатация неиспользованного оборудования;
  \item неправильное применение электронагревательных приборов;
  \item неисправность проводки;
  \item короткое замыкание;
  \item нарушение правил пожарной безопасности.
\end{itemize}

Для отвода тепла от персонального компьютера необходимы работающие вентиляторы, помещение проветривается, поэтому кислород, как окислитель процессов горения, имеется в достаточном количестве.
Исходя из этого, помещение кабинета, согласно нормам СП 2.13130.2012 <<Системы противопожарной защиты. Обеспечение огнестойкости объектов защиты>>, по степени пожаробезопасности следует отнести к категории Д (помещения, в которых в обращении находятся негорючие вещества и материалы в холодном состоянии).

В качестве средств тушения пожара применяются углекислотные огнетушители, используемые для тушения электроустановок, находящихся под напряжением.

\subsection{Электробезопасность}

В кабинет электроэнергия поступает для питания персональных компьютеров и электрического освещения.
Питание осуществляется от трехфазной сети переменного тока напряжением 380/220~В (+10..-15\%) частотой 50~Гц (+1~Гц).

Поскольку помещение сухое (относительная влажность не более 75\%), температура не превышает 30~°С, то, согласно ПУЭ 7 (<<Правилам устройства электроустановок>>), оно не относится к категории помещений повышенной опасности.
Однако возможна потенциальная опасность поражения людей электрическим током.
Источниками и причинами опасности являются:
\begin{itemize}
  \item открытые токопроводящие части оборудования, кабельной проводки;
  \item неисправность электрооборудования, электрических розеток;
  \item короткое замыкание в результате повреждения изоляции.
\end{itemize}

Для предотвращения поражения электрическим током потребителей электроэнергии в кабинете необходимо предусмотреть следующие технические мероприятия:
\begin{itemize}
  \item все токопроводящие части оборудования и кабельной проводки должны быть защищены ограждающими кожухами;
  \item все металлические конструкции, которые могут оказаться под напряжением в результате короткого замыкания, должны быть заземлены, защищены и выполнено защитное   отключение.
\end{itemize}

В качестве заземляющих проводников должны быть использованы элементы металлических конструкций, металлическое обрамление кабельных каналов.
Здание должно быть оборудовано комплексом мер, предотвращающих попадание энергии молнии в электрическую сеть, а также поражение людей, для чего на здание устанавливаются громоотводы.

Кроме технических, необходимо проведение организационных мероприятий:
\begin{itemize}
  \item к работе с электроустановками допускаются только лица, прошедшие инструктаж и проверку знаний правил техники безопасности в соответствии с ГОСТ 12.1.009-76, ПТЭ и ПТБ;
  \item периодически осуществляется контроль сопротивления электрической изоляции токоведущих частей (в соответствии с требованиями ПУЭ 7, оно не должно быть ниже 0.5~мм по отношению к корпусу ЭВМ).
\end{itemize}

\subsection{Эргономика и техническая эстетика}

Эффективность работы администратора (программиста) во многом зависит от организации рабочих мест.
Рабочее положение администратора --- сидячее.
Стул по возможности должен быть регулируемым по высоте, поскольку клавиатура и дисплей компьютеров должны находиться в зоне наилучшего обзора.
Для сохранения работоспособности имеет большое значение выбор основной рабочей позы.

Техническая эстетика позволяет снижать нервное утомление и вредные воздействия на функции организма в процессе труда.
Огромное значение в эстетическом оформлении производства имеет цвет.
Окраска, форма, внешний вид производственного помещения и оборудования улучшают условия освещения, а также психологическое состояние человека. Стены имеют светло-зеленый цвет, не вызывающий раздражения, потолок --- белый цвет, что обеспечивает максимальное отражение света.

Рассматриваемое помещение соответствует требованиям ГОСТ 12.2.032-78 <<Рабочее место при выполнении работ сидя. Общие эргономические требования>>.

\subsection{Режим труда и отдыха}

Работа администратора относится к категории работ связанных с опасными и вредными условиями труда.
В процессе труда на администратора оказывают действие следующие опасные и вредные производственные факторы, физические:
\begin{itemize}
  \item повышенный уровень статического электричества;
  \item повышенный уровень шума;
  \item повышенные уровни запыленности воздуха рабочей зоны;
  \item повышенная яркость светового изображения;
  \item повышенный или пониженный уровень освещенности;
  \item неравномерность распределения яркости в поле зрения;
  \item повышенное значение напряжения в электрической цепи, замыкание которой может произойти через тело человека.
\end{itemize}

Рациональный режим труда и отдыха --- это правильное чередование работы и перерывов в ней в течение смены, суток, недели, года, устанавливаемое с целью обеспечения высокой производительности труда и сохранения здоровья работающих.
Основным перерывом является перерыв на обед.
Обеденный перерыв при 8-часовой рабочей смене устанавливается продолжительностью не менее 30~мин через 4 часа после начала работы.

Режим отдыха складывается из нескольких компонентов:
\begin{itemize}
  \item времени на гигиенические процедуры и личные надобности (2-3) от сменного времени независимо от вида труда;
  \item времени регламентированных перерывов на отдых (входят в состав рабочего времени), определяемого по показателю условий труда или по интегральному показателю снижения работоспособности;
  \item времени микропауз, а также времени обеденного перерыва (нерабочего времени), остающегося от приема пищи.
\end{itemize}

Режим труда и отдыха должен быть построен в соответствии с особенностями трудовой деятельности пользователей персонального компьютера и характером функциональных изменений со стороны различных систем организма работников.

\subsection{Выводы}

В ходе выполнения работы, была дана краткая характеристика помещения и выполняемых работ.
Составлен план помещения и размещения оборудования.
Были определены оптимальные параметры микроклимата, шума, освещения.
Даны рекомендации по эргономике и режиму труда.

Наиболее неблагоприятными факторами являются микроклимат и искусственное освещение, обеспечиваемое люминесцентными лампами.

\clearpage
