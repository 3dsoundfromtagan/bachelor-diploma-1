\section{Системный анализ виртуальной инфраструктуры}

На данный момент при анализе и синтезе сложных программных и аппаратных систем все чаще используется системный подход.
Важным моментом для системного подхода является определение структуры системы --- совокупности связей между элементами системы, отражающих их взаимодействие.
Совокупность элементов и связей между ними позволяет судить о структуре системы.

Принципы системного анализа --- это некоторые положения общего характера, являющиеся обобщением опыта работы человека со сложными системами.
Общепринятых формулировок на настоящее время нет, но не все формулировки так или иначе описывают одни и те же понятия.
Пренебрежение принципами при проектировании любой нетривиальной технической системы, непременно приводит к потерям того или иного характера, от увеличения затрат в процессе проектирования до снижения качества и эффективности конечного продукта.

Системный анализ выполнен в соответствии с \cite{sys-analyz}.

\subsection{Принцип конечной цели}

Принцип конечной цели --- это абсолютный приоритет конечной цели, он имеет несколько правил:
\begin{itemize}
  \item Для проведения системного анализа необходимо, в первую очередь, сформировать цель исследования, так как не полностью определенные цели влекут за собой неверные выводы;
  \item Анализ следует вести на базе уяснения основной цели, что позволит определить ее существенные свойства показателей качества и критериев оценки;
  \item При синтезе систем любая попытка изменения или совершенствования должна оцениваться относительно конечной цели.
\end{itemize}

При использовании данного принципа, разрабатываемая виртуальная инфраструктура будет рассматриваться в виде <<черного ящика>>, функционирование которого описывается формулой \ref{blackbox-eq}.
\begin{equation} \label{blackbox-eq}
\text{Y=F(X, Z)}
\end{equation}
где Y --- выходной вектор системы, который функционально зависит от входного вектора X и вектора внутреннего состояния системы Z (рис. \ref{blackbox-pic}).
Входными данными (вектор X) будут являться запросы пользователей на обработку инцидентов и запросов на обслуживание.
Внутреннее состояние системы (вектор Z) представляет собой создание услуг по требованию клиента.
Выходными данными (вектор Y) будут являться предоставление доступа к услугам клиентам.
\addimghere{blackbox}{0.6}{Проектируемая система в виде черного ящика}{blackbox-pic}

В соответствии с данным принципом должна быть четко сформулирована конечная цель --- назначение проектируемой системы и сформирован список функций, которые должна выполнять система.
Кроме того, необходимо определить перечень входных воздействий, на которые реагирует система.

Цель проектирования --- разработка виртуальной инфраструктуры для реализации облачных услуг.
Список функций проектируемой системы:
\begin{itemize}
  \item Ф1 --- прием и регистрация обращений пользователей;
  \item Ф2 --- идентификация и обработка инцидентов и запросов на обслуживание;
  \item Ф3 --- создание, смена, обновление и удаление услуг по требованию;
  \item Ф4 --- предоставление доступа к услугам;
  \item Ф5 --- мониторинг состояния инфраструктуры.
\end{itemize}

Перечень входных воздействий на систему:
\begin{itemize}
  \item Внесение изменений данных о заявке;
  \item Мониторинг хода решения.
\end{itemize}



\clearpage
