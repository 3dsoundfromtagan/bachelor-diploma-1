\section{Постановка задачи}

Конечная цель проектирования --- разработка виртуальной инфраструктуры для реализации облачных услуг.

Виртуальная инфраструктура должна обладать следующими характеристиками:
\begin{itemize}
    \item использование, по возможности, продуктов, распространяющихся под свободной лицензией (GNU GPL) для организации инфраструктуры;
    \item устранение единой точки отказа при проектировании инфраструктуры;
    \item защита от распределенных атак на отказ (DDoS);
    \item использование инфраструктуры в бизнесе для предоставления облачных услуг клиентам.
\end{itemize}

Аппаратное обеспечение должно базироваться в дата-центре ориентированным на требования стандарта Tier III, состоящего из информационной, телекоммуникационной и инженерной инфраструктуры, с возможностью аренды выделенных серверов и аппаратной защиты от DDoS-атак, а также поддержкой аппаратного RAID.
Рекомендуемые характеристики выделенных серверов виртуализации:
\begin{itemize}
    \item 2xIntel\textregistered~Xeon E5430 @ 2.66GHz;
    \item минимальный объем ОЗУ 32 Гб;
    \item минимум 960 Гб места на жестком диске (SSD);
    \item поддержка аппаратного RAID;
    \item операционная система не ниже Debian 7 GNU/Linux или CentOS 6.5;
    \item интернет-канал с пропускной способностью не менее 100 Мб/с;
    \item возможность добавления дополнительных IP-адресов к серверам;
    \item возможность удаленного доступа к серверам посредством IPMI (Intelligent Platform Management Interface).
\end{itemize}

Следует предусмотреть возможность развертывания дополнительных виртуальных серверов, для организации DNS-серверов, системы мониторинга и платежной системы.
Также необходимо предусмотреть наличие системы хранения данных (СХД) для резервных копий, а также расширение дискового пространства на выделенных серверах.

Для разработки виртуальной инфраструктуры необходимо реализовать следующие этапы:
\begin{itemize}
    \item выбор технологий виртуализации;
    \item выбор физических серверов на основе услуги IaaS в ДЦ;
    \item приобретение лицензий на программное обеспечение;
    \item приобретение подсетей IP-адресов;
    \item создание и подключение инфраструктуры мониторинга, платежной системы (биллинга), резервного копирования;
    \item реализация аппаратной защиты от DDoS-атак;
    \item выбор перечня PaaS-услуг;
    \item внедрение перечня предоставляемых PaaS-услуг;
    \item создание руководства администратора по виртуализации;
    \item создание руководства для клиентов по часто задаваемым вопросам;
    \item внедрение инфраструктуры на рынке PaaS-услуг.
\end{itemize}

\clearpage
