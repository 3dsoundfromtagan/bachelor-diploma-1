\section{Постановка задачи}
Согласно приказу по ВУЗу №118-П от 7 марта 2015\footnote{Уточнить!!!} года была определена тема дипломного проекта бакалавра.
Тема проекта: <<Разработка виртуальной инфраструктуры для реализации облачных услуг>>.

Конечная цель проектирования --- разработка виртуальной инфраструктуры для реализации облачных услуг.

Виртуальная инфраструктура должна обладать следующими характеристиками:
\begin{itemize}
    \item Использование продуктов, распространяющихся под свободной лицензией (GNU GPL) для организации инфраструктуры;
    \item Устранение единой точки отказа при проектировании инфраструктуры;
    \item Защита от распределенных атак на отказ (DDoS);
    \item Использование инфраструктуры в бизнесе для предоставления облачных услуг клиентам.
\end{itemize}

Аппаратное обеспечение должно базироваться в ДЦ ориентированным на требования стандарта TIA-942, состоящий из информационной, телекоммуникационной и инженерной инфраструктуры, с возможностью аренды выделенного сервера и аппаратной защиты от DDoS-атак, а также поддержкой аппаратного RAID.
Минимальные характеристики выделенного сервера виртуализации:
\begin{itemize}
    \item 8xIntel\textregistered~Xeon E5430 @ 2.66GHz;
    \item Минимальный объем ОЗУ 32 Гб;
    \item Минимум 1 Тб места на жестком диске;
    \item Аппаратный RAID 1;
    \item ОС Debian 7 GNU/Linux или CentOS 6.5 и выше;
    \item Интернет-канал с пропускной способностью не менее 100 Мб/с;
    \item Возможность добавления дополнительных IP-адресов к серверу;
    \item Возможность удаленного доступа к серверу посредством IPMI.
\end{itemize}

Следует предусмотреть возможность развертывания дополнительных виртуальных серверов, для организации DNS-серверов и систем мониторинга.
Также необходимо предусмотреть наличие системы хранения данных (СХД) для резервного копирования.

Для разработки виртуальной инфраструктуры необходимо реализовать следующие этапы:
\begin{itemize}
    \item Выбор технологии виртуализации;
    \item Выбор физического сервера на основе услуги IaaS в ДЦ;
    \item Приоборетение лицензий на ПО;
    \item Приобретение подсетей IP-адресов;
    \item Создание и подключение инфраструктуры мониторинга, платежной системы (биллинга), резервного копирования;
    \item Реализация аппаратной защиты от DDoS-атак;
    \item Выбор перечня PaaS-услуг;
    \item Внедрение перечня предоставляемых PaaS-услуг;
    \item Создание руководства администратора по виртуализации;
    \item Создание руководства для клиентов по часто задаваемым вопросам;
    \item Внедрение инфраструктуры на рынке PaaS-услуг.
\end{itemize}

\clearpage
