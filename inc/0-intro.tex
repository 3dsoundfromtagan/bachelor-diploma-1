\anonsection{Введение}

Облачные услуги --- это способ предоставления, потребления и управления технологией.
Данный тип услуг выводит гибкость и эффективность на новый уровень, путем эволюции способов управления, таких как непрерывность, безопасность, резервирование и самообслуживание, которые соединяют физическую и виртуальную среду.
В данной сфере возрастает потребность в качественно продуманной архитектуре, позволяющей надежно и правильно организовать облачную инфраструктуру.

Для эффективной работы облачной инфраструктуры требуется эффективная структура и организация.
Небольшая команда из специалистов и бизнес-пользователей может создать обоснованный план и организовать свою работу в инфраструктуре.
Данная выделенная группа может намного эффективнее построить и управлять нестандартной облачной инфраструктурой, чем если компании будут просто продолжать добавлять дополнительные сервера и сервисы для поддержки центра обработки данных (ЦОД).

IaaS (Infrastructure as a Service) --- это предоставление пользователю компьютерной и сетевой инфраструктуры и их обслуживание как услуги в форме виртуализации, то есть виртуальной инфраструктуры.
Другими словами, на базе физической инфраструктуры дата-центров (ДЦ) провайдер создает виртуальную инфраструктуру, которую предоставляет пользователям как сервис.
Стоит отметить, что IaaS не предполагает передачи в аренду программного обеспечения, а всего лишь предоставляет доступ к вычислительным мощностям.

Технология виртуализации ресурсов позволяет физическое оборудование (сервера, хранилища данных, сети передачи данных) разделить между пользователями на несколько частей, которые используются ими для выполнения текущих задач.
К примеру, на одном физическом сервере можно запустить сотни виртуальных серверов, а пользователю для решения задач выделить время доступа к ним.

\clearpage
